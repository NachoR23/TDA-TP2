\section{Ejemplos ilustrativos del algoritmo de optimización de ataques}

En los próximos ejemplos presentamos casos límite donde se muestra que la estrategia del algoritmo siempre da la solución óptima.

\subsection{Ejemplo 1: Todos los ataques con la misma potencia}

\paragraph{Planteo del problema.}  
Supongamos que en todos los minutos la potencia acumulada disponible es la misma:
\begin{verbatim}
x_i (enemigos por minuto) : 100 150 200 250
f_i (ataque máximo por carga) : 300 300 300 300
\end{verbatim}
La meta es maximizar la cantidad de tropas eliminadas usando ataques óptimos y cargando cuando convenga.

\paragraph{Aplicación del algoritmo.}  
Como todos los valores de $f_i$ son iguales, cualquier minuto de ataque elimina la misma cantidad de enemigos siempre que la potencia acumulada alcance los enemigos presentes.  
El algoritmo DP selecciona ataques en los minutos donde $x_i$ sea mayor primero, acumulando la máxima cantidad de tropas eliminadas.

\paragraph{Resultado esperado.}  
Si atacamos en los minutos con $x_i$ más grandes primero, la cantidad total de tropas eliminadas es máxima.  
El algoritmo produce la secuencia:
\[
\text{Atacar: minuto 4, minuto 3, minuto 2, minuto 1}
\]
y elimina un total de tropas:
\[
\text{Total tropas eliminadas} = \min(250,300) + \min(200,300) + \min(150,300) + \min(100,300) = 700
\]

\paragraph{Empates y flexibilidad.}  
Si algunos minutos tienen la misma cantidad de enemigos, por ejemplo $x_2 = x_3 = 150$, el algoritmo puede elegir cualquiera de esos minutos sin afectar el total máximo de tropas eliminadas.

\subsection{Ejemplo 2: Incremento progresivo de potencia y enemigos}

\paragraph{Planteo del problema.}  
Considere la siguiente entrada:
\begin{verbatim}
x_i : 50 100 150 200
f_i : 50 100 150 200
\end{verbatim}
Cada minuto, los enemigos que llegan y la potencia de ataque disponible aumentan de manera proporcional.

\paragraph{Aplicación del algoritmo.}  
El algoritmo DP calcula para cada minuto si conviene atacar o cargar.  
- Atacar demasiado temprano limita el aprovechamiento de la potencia acumulada.  
- Atacar demasiado tarde hace que los enemigos se acumulen sin poder eliminarlos.  

\paragraph{Resultado esperado.}  
La estrategia óptima es atacar en cada minuto usando la potencia exacta necesaria:
\[
\text{Secuencia de decisiones} = \text{Atacar en minutos 1, 2, 3, 4}
\]
Cantidad de tropas eliminadas:
\[
50 + 100 + 150 + 200 = 500
\]

\paragraph{Comparación con un orden subóptimo.}  
Si el algoritmo eligiera atacar solo en los dos primeros minutos y cargar en los últimos, se eliminarían menos tropas:
\[
50 + 100 = 150 < 500
\]
Esto muestra que romper la secuencia óptima genera un total menor.

\paragraph{Empates y decisiones equivalentes.}  
Si dos minutos consecutivos tienen $x_i$ iguales y $f_i$ iguales, intercambiar ataques entre ellos no cambia el total de tropas eliminadas, lo que confirma que el algoritmo maneja correctamente los empates.
