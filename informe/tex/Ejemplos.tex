\section{Ejemplos ilustrativos del algoritmo greedy}
En los proximos dos ejemplos vamos a presentar casos border en donde se muestra que el algoritmo siempre nos da la solucion óptima

\subsection{Ejemplo 1: Todos los tiempos iguales}

\paragraph{Planteo del problema.}  
Tenemos la siguiente lista de batallas:
\begin{verbatim}
T_i,B_i
10,100
10,90
10,90
10,50
10,20
10,10
\end{verbatim}
Todas las batallas duran lo mismo ($t=10$). El objetivo es minimizar $\sum b_i F_i$.

\paragraph{Cálculo de razones y orden greedy.}  
Como $t_i=10$ para todas, la razón $\tfrac{b}{t}$ es proporcional al peso $b$.  
Ordenando de mayor a menor: 
\[
(10,100)\;\to\;(10,90)\;\to\;(10,90)\;\to\;(10,50)\;\to\;(10,20)\;\to\;(10,10).
\]

\paragraph{Cálculo del costo óptimo.}  
Los tiempos acumulados son $F=(10,20,30,40,50,60)$.  
Multiplicando cada $F_i$ por $b_i$:
\[
100\cdot10=1000,\;\;90\cdot20=1800,\;\;90\cdot30=2700,\;\;50\cdot40=2000,\;\;20\cdot50=1000,\;\;10\cdot60=600.
\]
La suma da:
\[
\sum b_i F_i = \boxed{9100}.
\]

\paragraph{Comparación con un orden incorrecto.}  
Si cambiamos el orden y ponemos primero $(10,90)$ y después $(10,100)$, obtenemos:
\[
\sum b_i F_i = 9200.
\]
Este valor es mayor que $9100$, mostrando que violar la regla empeora la solución.

\paragraph{Empates.}  
Las dos batallas $(10,90)$ tienen la misma razón $\tfrac{b}{t}=9$.  
Si las permutamos, el resultado total sigue siendo $9100$.  
Esto confirma que, ante empates, cualquier orden dentro del bloque es igualmente óptimo.

\subsection{Ejemplo 2: Empates no triviales en la razón}

\paragraph{Planteo del problema.}  
La lista de batallas es:
\begin{verbatim}
T_i,B_i
1,3
2,4
3,6
4,8
5,5
\end{verbatim}
Aquí los tiempos son distintos, y hay batallas con la misma razón $b/t$.

\paragraph{Cálculo de razones y orden greedy.}  
Las razones son:
\[
(1,3)\mapsto 3.0,\quad (2,4)\mapsto 2.0,\quad (3,6)\mapsto 2.0,\quad (4,8)\mapsto 2.0,\quad (5,5)\mapsto 1.0.
\]
El orden óptimo es:
\[
(1,3)\;\to\;(2,4)\;\to\;(3,6)\;\to\;(4,8)\;\to\;(5,5).
\]

\paragraph{Cálculo del costo óptimo.}  
Los tiempos acumulados son $F=(1,3,6,10,15)$.  
Contribuciones:
\[
3\cdot1=3,\;\;4\cdot3=12,\;\;6\cdot6=36,\;\;8\cdot10=80,\;\;5\cdot15=75.
\]
La suma da:
\[
\sum b_i F_i = \boxed{206}.
\]

\paragraph{Comparación con un orden incorrecto.}  
Si movemos $(5,5)$ delante de las batallas con razón $2.0$, obtenemos $F=(1,6,8,11,15)$ y un total de
\[
\sum b_i F_i = 251.
\]
Este valor es mayor que $206$, mostrando que romper el orden por $\tfrac{b}{t}$ empeora la solución.

\paragraph{Empates.}  
Las batallas $(2,4),(3,6),(4,8)$ tienen la misma razón $2.0$.  
Si se permutan entre ellas, por ejemplo $(4,8)\to(3,6)\to(2,4)$, el costo total sigue siendo $206$.  
Esto confirma que ante empates, cualquier permutación interna dentro del bloque mantiene la solución óptima.
