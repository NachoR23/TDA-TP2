\section{Análisis del Problema}

\subsection{Nomenclatura Utilizada}
Para simplificar y mejorar la explicación del problema, utilizaremos la siguiente nomenclatura:
\begin{itemize}
\item $n$: duración de la batalla.
\item $x_i$: cantidad de soldados enemigos que llegan en el minuto $i$.
\item $t_i$: duración de la batalla $i$.
\item $x_1, x_2, \cdots, x_n$: secuencia de ataques.
\item $f(\cdot)$: función que determina la cantidad de bajas enemigas si utilizamos la habilidad en un cierto momento.
\item $f(j)$: cantidad de enemigos eliminados transcurridos $j$ minutos.
\end{itemize}

\subsection{Planteamiento del Problema}
Para poder dar una solución, primero debemos comprender lo que se nos solicita. El objetivo es desarrollar un algoritmo mediante \textbf{programación dinámica} que determine el momento en el cual debemos atacar para garantizar la mayor cantidad de bajas enemigas.

\subsection{Herramientas para Resolver el Problema}
Una vez entendido el problema, debemos identificar la herramienta adecuada para su resolución. Se nos pide específicamente un algoritmo por programación dinámica, lo que corresponde a una estrategia que busca dividir un problema complejo en partes más simples y relacionadas, almacenando las soluciones de los subproblemas para no recalcularlas nuevamente.\\ 
Las bibliotecas para la medición de tiempos de ejecución son módulos o conjuntos de funciones predesarrollados por otros programadores que nos simplifican la tarea de cuantificar cuánto tarda una función o script en ejecutarse.
