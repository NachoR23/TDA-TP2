\section{Análisis del Problema}
\subsection{Nomenclatura Utilizada}
Para simplificar y mejorar la explicación del problema, utilizaremos la siguiente nomenclatura:
\begin{itemize}
\item $n$: duracion de la batalla.
\item $x_i$: cantidad de soldados enemigos que llegan en el minuto $i$.
\item $t_i$: duración de la batalla $i$.
\item  $x_1, x_2, \cdots, x_n$: secuencia de ataques.
\item $f(\cdot)$: funcion que determina la cantidad de bajas enemigas si utilizamos la habilidad en un cierto tiempo.
\item $f(j)$: cantidad de enemigos eliminados transcurridos j minutos.
\end{itemize}

\subsection{Planteamiento del Problema}
Para poder dar una solución, primero debemos comprender lo que se nos solicita. El objetivo es desarrollar un algoritmo por \textbf{programación dinámica} que encuentre el tiempo en el cual debemos atacar para poder garantizar la mayor cantidad de bajas enemigas.
\subsection{Herramientas para resolver el Problema}
Una vez entendido el problema, debemos identificar la herramienta adecuada para su resolución. Se nos pide específicamente un algoritmo por programación dinámica, lo que corresponde a una estrategia que busca dividir un problema complejo en partes más simples, que estén relacionadas, guardando las soluciones más simples para no recalcularlas de nuevo.\\ 
Las bibliotecas para medición de tiempos de ejecución son módulos o conjuntos de funciones predesarrollados por otros programadores que nos simplifican la tarea de cuantificar cuánto tarda una función o script en ejecutarse.