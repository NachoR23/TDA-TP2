\section{Empezamos a desarrollar la solución}
Para poder desarrollar una solución, lo que tenemos que plantear es la ecuación de recurrencia. Para eso empezamos reconociendo los casos base y las situaciones comúnes que aparecen.
Nomenclatura utilizada:
\begin{itemize}
\item $X[i]$:  bajase en un determinado minuto i.
\item $F[j]$: bajas si cargamos durante j minutos.
\item $n$: minutos.
\item $dp[i]$: bajas acumuladas hasta un minuto i.
\item $k$: minuto donde empezo la carga.
\end{itemize}
Caso base:
\begin{lstlisting}[language=Python]
    dp[0] = 0  
    # No hay minutos, no hay daño
\end{lstlisting}
Primer caso:
\begin{lstlisting}[language=Python]
    dp[i] = dp[i-1]
    # Decidimos no atacar en el minuto i
\end{lstlisting}
Segundo caso:
\begin{lstlisting}[language=Python]
    dp[i] = dp[k-1] + min(X[i], F[j])
    # Decidimos atacar en el minuto i
\end{lstlisting}
Lo que me piden es maximizar la cantidad de bajas enemigas, o sea podemos plantear la siguiente ecuación:
\begin{lstlisting}[language=Python]
    dp[i] = max(
        # Opcion 1: No atacar en minuto i
        dp[i-1],  
        # Opción 2: Atacar en i con ultimo ataque en k
        max_{k=1 to i} { dp[k-1] + min(X[i], F[i-k+1]) }
        )
\end{lstlisting}



