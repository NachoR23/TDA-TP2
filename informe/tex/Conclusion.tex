\section{Conclusión}
En este informe se buscó demostrar que existe una solución óptima al problema que enfrenta el Señor del Fuego y su ejército al momento de organizar sus batallas. El objetivo fue encontrar una forma de planificar los enfrentamientos de manera que la suma ponderada de las batallas se minimice. \\

Para resolver el problema se empleó una estrategia \textit{Greedy}, que permitió establecer un balance entre las batallas considerando tanto su tiempo de duración como su peso relativo. El camino hacia esta solución no fue inmediato: se exploraron distintos algoritmos que priorizaban otros aspectos, pero finalmente se arribó a esta propuesta, que se verificó como la más adecuada mediante un método directo. \\

Durante el desarrollo del algoritmo se analizó el impacto de los valores asignados a cada batalla en el tiempo de ejecución, además de estudiar la complejidad algorítmica de la solución. Como resultado, se obtuvo que la complejidad es $O(n \log n)$, lo cual fue corroborado mediante la técnica de cuadrados mínimos. \\

Finalmente, el algoritmo fue probado tanto con los casos de prueba proporcionados por la cátedra como con pruebas adicionales diseñadas por nosotros, incluyendo casos borde. Los resultados confirmaron la efectividad de la estrategia utilizada, validando así la pertinencia del enfoque adoptado. Este trabajo no solo permitió resolver el problema planteado, sino también afianzar el aprendizaje sobre el diseño, análisis y validación de algoritmos eficientes.
